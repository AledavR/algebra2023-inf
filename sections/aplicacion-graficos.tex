\section[Aplicación en procesamiento de imágenes y visión por computadora]{Aplicación en procesamiento de imágenes \\ y visión por computadora} % (fold)
\label{sub:Aplicación en procesamiento de imágenes y visión por computadora}

Los autovalores y autovectores se utilizan para reducir la dimensionalidad de los datos. Esto es útil para comprimir imágenes o para reconocer patrones en imágenes.

Los autovalores y autovectores son clave en el Análisis de Componentes Principales (PCA), que es una técnica de reducción de dimensionalidad,si  tienes una imagen de alta resolución. Esta imagen puede tener millones de píxeles y cada píxel puede ser una característica. 
PCA te permite reducir el número de características, manteniendo las que contienen la mayor parte de la información.


\subsection{Transformación de Imagenes} % (fold)
\label{sec:Transformación de Imagenes}

Las imágenes digitales se pueden representar como matrices, donde cada entrada en la matriz corresponde a un píxel en la imagen. Al aplicar una matriz de transformación a esta matriz de imagen, puedes cambiar la imagen de varias maneras, como escalarla, rotarla o inclinarla. Los autovectores de la matriz de transformación te dicen las direcciones en las que la imagen se estira o se encoge, y los autovalores te dicen cuánto se estira o se encoge la imagen en esas direcciones.
% subsubsection Transformación de Imagenes (end)

\subsection{Reducción de dimensionalidad} % (fold)
\label{sec:Reducción de dimensionalidad}
En el procesamiento de imágenes, a menudo tienes que lidiar con una gran cantidad de datos. Por ejemplo, una imagen de 1000x1000 píxeles tiene un millón de características si consideras cada píxel como una característica. Sin embargo, no todas estas características son útiles o importantes. Aquí es donde entran en juego los autovalores y autovectores. Al calcular los autovalores y autovectores de la matriz de covarianza de los datos de la imagen, puedes identificar las características más importantes de la imagen. Este proceso se conoce como Análisis de Componentes Principales (PCA).
% subsection Reducción de dimensionalidad (end)

\subsection{Reconocimiento de patrones} % (fold)
\label{sec:Reconocimiento de patrones}
Los autovalores y autovectores también se utilizan en algoritmos de reconocimiento de patrones, como el Reconocimiento de Caras Eigenface. En este algoritmo, los autovectores se utilizan para definir un conjunto de ``caras base'', y cualquier cara en la base de datos.
% subsection Reconocimiento de patrones (end)



% section Aplicación en procesamiento de imágenes y visión por computadora (end)
