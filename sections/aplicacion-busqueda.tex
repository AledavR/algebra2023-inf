\section{Algoritmo patentado para la clasificación de paginas web} % (fold)
\label{sub:Algoritmo patentado para la clasificacion de paginas web}


PageRank es una familia de algoritmos creada y desarrollada por la compañía tecnológica estadounidense Google para optimizar las búsquedas de páginas web. Patentado el 9 de enero de 1999, es la base lógica sobre la que se fundamenta su motor de búsqueda, que en poco tiempo se impondría a todos sus competidores, incluidos AltaVista y Yahoo! Fue desarrollado por Larry Page (apellido del que toma su nombre) y Serguéi Brin.

El algoritmo inicial del PageRank lo podemos encontrar en el documento original donde sus creadores presentaron el prototipo de Google: \textit{The Anatomy of a Large-Scale Hypertextual Web Search Engine}

\[
  \text{PR}(A) = (1-d) + d \displaystyle\sum_{i=1}^{n}\frac{\text{PR}(i)}{C(i)}
\]
% subsection Algoritmo patentado para la clasificación de paginas web (end)
