\section{Introducción} % (fold)
\label{sec:Introducción}

El álgebra lineal como cualquiera de las diferentes áreas que forman parte de el estudio matemático conforma la base para el desarrollo de herramientas que sirven para realizar un estudio consciente de la realidad. Es este sentido podemos encontrar distintos aspectos de la biología en los cuales los conceptos que definen el álgebra lineal juegan un papel fundamental en su comprensión.

Llamamos un sistema evolutivo a aquel que varia en el tiempo de acuerdo a diferentes factores. Los modelos biológicos definen una clase de los ya mencionados sistemas evolutivos, estos modelos nos permiten realizar un mapeado de los diferentes factores que afectan a los seres vivos que conforman el sistema. Podemos encontrar modelos biológicos en escenarios como el estudio de la evolución cuantitativa de una población, la distribución de los estados de los individuos en una pandemia o la distribución genética de una población. El álgebra lineal brinda conceptos como los sistemas matriciales o los autovalores, los cuales permiten encontrar una solución a los diferentes sistemas evolutivos que se puedan presentar. 

El siguiente informe definirá algunos de estos conceptos matemáticos en mayor detalle para luego adentrarnos en la definición, clasificación y construcción de los diferentes modelos biológicos, además de brindar algunos ejemplos puntuales de la aplicación para una correcta ilustración del tema.

% section Introducción (end)
