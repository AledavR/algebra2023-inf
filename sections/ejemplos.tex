\section{Ejemplos de aplicación}

\subsection{Caso contagio de enfermedad}
Se busca analizar el desarrollo de una cepa de una enfermedad contagiosa en una población de 1000 individuos. Se conoce que luego de una semana un individuo sano posee un 70\% de probabilidades de contraer la enfermedad, además se conoce que luego de una semana un individuo enfermo tiene un 40\% de probabilidades de curarse sin obtener inmunidad a la enfermedad, un 10\% de probabilidad de fallecer a causa de la enfermedad y un 40\% de probabilidad de desarrollar inmunidad permanente a la enfermedad. Se busca conocer la distribución de la población luego de cinco semanas de haber comenzado el contagio.

Comenzaremos el análisis observando los posibles estados en los que se puede encontrar un individuo:

\begin{itemize}
    \item Sano
    \item Enfermo
    \item Inmunizado
    \item Muerto
\end{itemize}

Además se tiene que el vector de población inicial sera igual a:

\[
    \vec{p} = 
    \left[
    \begin{array}{c}
        1000\\
        0\\
        0\\
        0
    \end{array}
    \right]
\]

Conocemos las probabilidades que definen los cambios de estado de un ejemplar sano y enfermo. Los estados restantes no presentaran una variación en el tiempo. 



Por lo tanto el modelo del desarrollo de la enfermedad sera de la siguiente manera:

\begin{figure}[htb]
\centering
\begin{tikzpicture}
\node[state] (s1) {Sano};
\node[state, right of=s1] (s2) {Enfermo};
\node[state, below left of=s2] (s3) {Inmune};
\node[state, below right of=s2] (s4) {Muerto};

\draw (s1) edge[loop left] node {0.3}  (s1);
\draw (s1) edge[bend left] node {0.7}  (s2);


\draw (s2) edge[bend left=12] node {0.4} (s1);
\draw (s2) edge[loop right] node {0.1} (s2);
\draw (s2) edge[bend right=12] node{0.4} (s3);
\draw (s2) edge[bend left=12] node{0.1} (s4);

\draw (s3) edge[loop left] node {1} (s3);
\draw (s4) edge[loop right] node{1} (s4);

\end{tikzpicture}
\caption{Cadena de Markov}

\end{figure}

De donde podemos obtener la siguiente matriz de transición:

\[
    A = 
    \left[
        \begin{array}{cccc}
            0.3 & 0.4 & 0 & 0 \\
            0.7 & 0.1 & 0 & 0 \\
            0 & 0.4 & 1 & 0 \\
            0 & 0.1 & 0 & 1
        \end{array}
    \right]
\]
Podemos observar que la matriz solo posee valores reales positivos y además los elementos de cada columna suman 1 entonces se puede decir que la matriz de transición es una matriz estocástica. Entonces por el teorema \ref{esto1} podemos garantizar que $\lambda = 1$ será un autovalor de la matriz.

Con ayuda de una herramienta computacional se obtienen los autovalores y autovectores asociados a la matriz de transición:

\begin{gather*}
    \lambda_1,\lambda_2 = 1 \quad E_{\lambda_1,\lambda_2} = L\{(0,0,1,0)^t,(0,0,0,1)^t\}\\
    \lambda_3 = 0.738516 \quad E_{\lambda_3} = L\{(-2.38516,-2.6148,4,1)^t\}\\
    \lambda_4 = -0.338516 \quad E_{\lambda_4} = L\{(8.38516,-13.3852,4,1)^t\}\\
\end{gather*}

Por la proposición \ref{autoli} se sabe que los autovectores asociados a autovalores diferentes serán linealmente independientes por lo tanto el conjunto $B = \{v_1,v_2,v_3,v_4\}$ sera una base de espacio vectorial $\mathbb{R}^4$ donde $v_i$ sera el autovector asociado al autovalor $\lambda_i, i = 1,2,3,4$

Como la multiplicidad algebraica de cada autovalor coincide con la multiplicidad geométrica. Entonces la matriz $A$ sera diagonalizable de la siguiente forma:

\[
    A = PDP^{-1}
\]

Luego si se quiere conocer la distribución de la población luego de 5 semanas se hará uso del teorema \ref{diagpot} de las matrices diagonalizables:

\[
    A^5\vec{p} = PD^5P^{-1}\vec{p}
\]

Como el vector de población inicial pertenece a $R^4$ este podrá ser escrito de la siguiente forma:

\[
    \vec{p} = w_1v_1 + w_2v_2 + w_3v_3 + w_4v_4
\]
 y la matriz $P^{-1}$ describirá una matriz de cambio de base de la base canónica a la base $B$. Entonces podemos desarrollar la expresión $A^5\vec{p} = PD^5P^{-1}\vec{p}$ de  la siguiente manera

 \begin{gather*}
    A^5\vec{p} = P D^5 \left[\begin{array}{c} w_1\\ w_2\\ w_3\\ w_4 \end{array}\right]\\
    A^5\vec{p} = P \left[\begin{array}{c} \lambda_1^5w_1\\ \lambda_2^5w_2\\ \lambda_3^5w_3\\ \lambda_4^5w_4 \end{array}\right]\\
    A^5\vec{p} = \lambda_1^5w_1v_1 + \lambda_2^5w_2v_2 + \lambda_3^5w_3v_3 + \lambda_4^5w_4v_4\\
 \end{gather*}

 Hallando los valores $w_i$:

 \[
    \vec{p} = 800v_1 + 200v_2 - 248.558 v_3 + 48.556 v_4
 \]

 Finalmente la expresión $A^5\vec{p}$ queda de la siguiente forma:

 \[
        A^5\vec{p} = 1^5(800)v_1 + 1^5(200)v_2 + 0.738516^5(-248.558)v_3 -0.338516^5(48.556)v_4
 \]

 Opererando:

 \[
    A^5\vec{p} = (0,0,800,0) + (0,0,0,200) + (130,142,-218,-54) + (-2,3-1,0)\\
\]


Obtenemos que luego de 5 semanas la población presentara la siguiente distribución:
\[
    A^5\vec{p} = (128,145,581,146)
\]
 teniendo a 128 individuos sanos, 145 enfermos,581 que han desarrollado inmunidad y 146 que han perecido a causa de la enfermedad. Sumando los valores obtenemos 1000 que coincide con el numero de individuos en la población.

 Adicionalmente podemos hacer el estudio en un tiempo $n$ que tiende al infinito tal que:
\[
    \lim_{n \to \infty}A^n\vec{p} = \lim_{n \to \infty} \underbrace{1^n}_{=1}(800)v_1 + \underbrace{1^n}_{=1}(200)v_2 + \underbrace{0.738516^n}_{\approx 0}(-248.558)v_3 \underbrace{-0.338516^n}_{\approx 0}(48.556)v_4
\]
De donde obtenemos que:
\[
    A^n\vec{p} = (0,0,800,200)
\]
que quiere decir que cuando la enfermedad ya no se pueda seguir propagando habrán perecido 200 individuos a causa de esta y habrán aproximadamente 800 individuos con inmunidad.
 
\subsection{Caso crecimiento poblacional de ratones}
Se estudia a una especie de ratones donde se consideran 3 etapas vitales: crías, jóvenes y adultos. Esta especie de ratón progresa de cría a joven en un determinado periodo de tiempo, de joven a adulto en un periodo de tiempo similar y, finalmente, los jóvenes y adultos tienen crías en un periodo de tiempo similar a los anteriores. Se desea saber la distribución de la población luego de 10 periodos de tiempo, con una distribución inicial de 10 adultos, 15 jóvenes y 40 crías. Se conoce que solo el 20\% de las crías llegan a ser jovenes y solo un 50\% de los ratones jóvenes llegan a adultos, cada ratón joven tiene una media de 5 crías y que cada adulto tiene una media  3 de crías. 

Lo primero que debemos hacer es definir los posibles etapas de un individuo de la especie, en este caso se nos otorga de manera directa los cuales son:

\begin{itemize}
    \item Cría (C)
    \item Joven (J)
    \item Adulto (A)
\end{itemize}

además se tiene un vector de población inicial $\vec{p} = \left[ \begin{array}{c}
    40 \\
    15 \\
    10
\end{array} \right] $

Denotamos las siguientes expresiones:

\begin{align*}
    C_t &= \text{numero de crías en el instante de tiempo t}\\
    J_t &= \text{numero de jóvenes en el instante de tiempo t}\\
    A_t &= \text{numero de adultos en el instante de tiempo t}
\end{align*}

Luego definimos relaciones entre las diferentes etapas:
\[
\begin{cases}
    C_t = 5 J_{t-1} + 3 A_{t-1} \text{(Cada adulto tiene 3 crías y cada joven 5 crías)}\\
    J_t = 0.2 C_{t-1} \text{(20\% de las crías llegan a la juventud)}\\
    A_t = 0.5 L_{t-1} \text{(50\% de los jóvenes llegan a ser adultos)}\\
\end{cases}
\]

Por lo tanto podemos representar gráficamente al modelo de la siguiente forma:
\begin{figure}[htb]

\centering
\begin{tikzpicture}
\node[state] (s1) {C};
\node[state, right of=s1] (s2) {J};
\node[state, right of=s2] (s3) {A};

\draw (s1) edge[bend left] node {0.2}  (s2);
\draw (s2) edge[bend left] node {0.5}  (s3);


\draw (s2) edge[bend left=12] node {5} (s1);
\draw (s3) edge[bend left] node {3} (s1);

\end{tikzpicture}

\caption{Modelo de Leslie}
\end{figure}

Además se obtiene la matriz de transición ($A$):

\[
A =
\left[
    \begin{array}{ccc}
        0 & 5 & 3 \\
        0.2 & 0 & 0 \\
        0 & 0.5 & 0
    \end{array}
\right]
\]

Se sabe que la distribución de la población en el instante de tiempo $t$ es dependiente del instante de tiempo anterior $t-1$:

\[
    \left[
    \begin{array}{c}
        C_t \\
        J_t \\
        A_t
    \end{array}
    \right] = 
    \left[
    \begin{array}{ccc}
        0 & 5 & 3 \\
        0.2 & 0 & 0 \\
        0 & 0.5 & 0
    \end{array}
    \right]
    \left[
    \begin{array}{c}
        C_{t-1} \\
        J_{t-1} \\
        A_{t-1}
    \end{array}
    \right]
\]

Hallamos los autovalores y autovectores de la matriz $A$:

\begin{gather*}
    \lambda_1 = -0.3389; \quad v_1 = (1.149,-0.678,1)\\
    \lambda_2 = -0.7865; \quad v_2 = (6.186,-1.573,1)\\
    \lambda_3 = 1.1254; \quad v_3 = (12.666,2.251,1)\\
\end{gather*}

Como $v_1, v_2 \text{ y } v_3$ están asociados a tres autovalores distintos entonces son linealmente independientes y conforman una base $B$ del espacio $\mathbb{R}^3$.

Hallando los coeficientes de la combinación lineal de $\vec{p}$ en la base $B$:
\[
    \vec{p} = 11.281v_1 - 6.677v_2 + 5.396v_3
\]

Realizando el procedimiento visto en el anterior ejercicio obtenemos la siguiente expresión con la cual determinamos $A^t\vec{p}$:

\[
    A^t\vec{p} = (-0.3389)^t (11.281)v_1 + (-0.7865)^t(-6.677) v_2 + (1.1254)^t(5.396)v_3
\]

Como deseamos obtener la distribución luego de 10 periodos, reemplazamos $t$ con 10:


\begin{gather*}
   A^{10}\vec{p} = \underbrace{(-0.3389)^{10}}_{\approx 0} (11.281)v_1 + \underbrace{(-0.7865)^{10}}_{\approx 0}(-6.677) v_2 + (1.1254)^{10}(5.396) v_3\\
   A^{10}\vec{p} = (17.5849)v_3\\
\end{gather*}


Operando tenemos:

\[
\vec(p)_{10}  = 
\left[\begin{array}{c}
    223 \\
    36 \\
    18
\end{array}\right]
\]

Siendo $\vec{p}_{10}$ la representación vectorial de la distribución en el decimo periodo. Ademas podemos obtener la tasa de crecimiento de un periodo a traves de la expresión  \mbox{$|1 - \lambda_{d}|\times 100\%$} siendo esta cercana a $12.5\%$

Finalmente para un tiempo $t$ que tiende la infinito:

\[
	\lim_{t \to \infty} A^tE = \lim_{t \to \infty}\underbrace{\lambda_1^t}_{\approx 0} w_1v_1 + \underbrace{\lambda_2^t}_{\approx 0} w_2v_2+  \lambda^t w_3v_3
\]

En el caso $t \to \infty$ la distribución esta definida solo por

\[
    A^tE = \lambda^t_3 w_3v_3
\]

    
%\subsection{Caso Ecosistema}
