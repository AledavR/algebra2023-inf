\section{Conclusiones} % (fold)
\label{sec:Conclusiones}

En el informe pudimos definir algunos de los modelos discretos mas usados en el estudio del comportamiento de la distribución de las poblaciones a lo largo del tiempo, mostrando diferentes ejercicios para ilustrar los diferentes modelos, demostrando así la importancia de la abstracción matemática al momento de analizar sistemas basados en la realidad.

Como se vio a lo largo del informe, el area del álgebra relacionada al estudio de las matrices y sus propiedades proporcionan herramientas beneficiosas para el estudio en diferentes disciplinas, en este caso los sistemas biologicos, ya sean poblaciones singulares o un conjunto de ellas interactuando constantemente entre sí.

Pero lo visto no representa todo el potencial que el álgebra lineal puede brindar al estudio de sistemas dinámicos. El álgebra lineal es también usado en los modelos biológicos continuos, los cuales presentan un mayor grado de complejidad a la vez que nos permiten obtener mayor información sobre el ecosistema que se estudia. Pero esta clase de modelos biológicos necesitan el conocimiento además de las ecuaciones diferenciales ordinarias, tema que escapa del alcance que se busca dar al presente informe.

Finalmente, podemos decir que las matemáticas nos abren el camino a diferentes métodos de estudio del entorno que nos rodea.


% section Conclusiones (end)

