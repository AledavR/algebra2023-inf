

\documentclass[12pt]{article}
\usepackage[utf8]{inputenc}
\usepackage[spanish]{babel}
\usepackage[]{hyperref}
\usepackage[T1]{fontenc}
\usepackage{amsmath}
\usepackage{amsthm}
\usepackage{amsfonts}
%\usepackage[]{todonotes}
\usepackage[]{authblk}
\usepackage{amssymb}
\usepackage{multicol}
\usepackage{tikz-cd}
\usepackage{lipsum}	
\usepackage{mathtools}
\usepackage{mathrsfs}
\usepackage{tikz}
\usepackage[siunitx]{circuitikz}
\usepackage{stmaryrd}
%\usepackage[none]{hyphenat}
\usetikzlibrary{babel}
\usepackage[a4paper, margin=1in]{geometry}
\usepackage[shortlabels]{enumitem}
\usetikzlibrary{automata}
\usetikzlibrary{positioning}  %                 ...positioning nodes
\usetikzlibrary{arrows}       %                 ...customizing arrows
\tikzset{node distance=4.5cm, % Minimum distance between two nodes. Change if necessary.
         every state/.style={ % Sets the properties for each state
           semithick,
           fill=gray!10},
         initial text={},     % No label on start arrow
         double distance=4pt, % Adjust appearance of accept states
         every edge/.style={  % Sets the properties for each transition
         draw,
           ->,>=stealth',     % Makes edges directed with bold arrowheads
           auto,
           semithick}}


\title{Álgebra lineal aplicada al estudio de modelos biológicos discretos}

\author{Alejandro Ramirez Chero}
%\author[2]{Luis Enrique Huayta Huillcahuare}
%\author[3]{Luis Felipe Pozo Velarde}
%\affil[1,2,3]{Escuela de Computación Científica, Universidad Nacional Mayor de San Marcos}
\affil{Escuela de Computación Científica, Universidad Nacional Mayor de San Marcos}
%\date{19/09/2023}

\newenvironment{amatrix}[1]{%
	\left(\begin{array}{@{}*{#1}{c}|c@{}}
	}{%
	\end{array}\right)
}

\newenvironment{mtx}[1]{%
  \left[\begin{array}{@{}*{#1}{c}@{}}
	}{%
\end{array}\right]
}

\newcommand{\lmapdef}[3]{
  #1: #2 \rightarrow #3
}

\newcommand{\opnm}[1]{
  \operatorname{#1}
}

\newcommand{\bs}[1]{
  \mathcal{#1}
}


\newtheorem{definition}{Definición}
\newtheorem{theorem}{Teorema}[definition]
\newtheorem{prop}{Proposición}[definition]


\begin{document}



\maketitle

\begin{abstract}
	En el siguiente informe se realizará un breve análisis a los conceptos del álgebra lineal que  son usados en el estudio de sistemas de evolución. Luego se expondrán algunas definiciones sobre el modelamiento de sistemas biologicos haciendo principal énfasis en el análisis de modelos biológicos discretos, brindando además algunos ejemplos ilustrativos.
\end{abstract}

\pagebreak

\tableofcontents

% Contenido

\section{Introducción} % (fold)
\label{sec:Introducción}

El álgebra lineal como cualquiera de las diferentes áreas que forman parte de el estudio matemático conforma la base para el desarrollo de herramientas que sirven para realizar un estudio consciente de la realidad. Es este sentido podemos encontrar distintos aspectos de la biología en los cuales los conceptos que definen el álgebra lineal juegan un papel fundamental en su comprensión.

Llamamos un sistema evolutivo a aquel que varia en el tiempo de acuerdo a diferentes factores. Los modelos biológicos definen una clase de los ya mencionados sistemas evolutivos, estos modelos nos permiten realizar un mapeado de los diferentes factores que afectan a los seres vivos que conforman el sistema. Podemos encontrar modelos biológicos en escenarios como el estudio de la evolución cuantitativa de una población, la distribución de los estados de los individuos en una pandemia o la distribución genética de una población. El álgebra lineal brinda conceptos como los sistemas matriciales o los autovalores, los cuales permiten encontrar una solución a los diferentes sistemas evolutivos que se puedan presentar. 

El siguiente informe definirá algunos de estos conceptos matemáticos en mayor detalle para luego adentrarnos en la definición, clasificación y construcción de los diferentes modelos biológicos, además de brindar algunos ejemplos puntuales de la aplicación para una correcta ilustración del tema.

% section Introducción (end)

\include{"sections/antecedentes.tex"}
\section{Marco Teórico} % (fold)
\label{sec:Marco Teórico}

\subsection{Matrices}
\begin{definition}
La matriz es un arreglo bidimensional formado por números o símbolos los cuales se encuentran distribuidos en una forma rectangular a través de lineas verticales u horizontales las cuales son llamadas columnas o filas respectivamente.
\end{definition}

Las matrices resultan útiles por la capacidad de crear representaciones de sistemas de ecuaciones lineales o diferenciales. Por ejemplo podemos representar el siguiente sistema de ecuaciones:

\[
    \begin{cases}
        ax_1 + bx_2 = y_1\\
        cx_1 + dx_2 = y_2
    \end{cases}
\]

Como el siguiente sistema de matrices:

\[
\left[\begin{array}{cc}
    a & b \\
    c & y
\end{array}\right]
\left[\begin{array}{c}
    x_1\\
    x_2 
\end{array}\right] =
\left[\begin{array}{c}
    y_1\\
    y_2 
\end{array}\right]
\]

Adicionalmente podemos denotar este sistema matricial como:

\[
    A\cdot X = Y
\]

\subsubsection{Matriz diagonal}
\begin{definition}
Llamamos matriz diagonal a una matriz cuadrada que solo posee elementos diferentes a $0$ en la diagonal principal. 
\end{definition}

Por ejemplo tenemos la siguiente matriz diagonal $D$ de orden $n\times n$:

\[D =
  \begin{mtx}{4}
    d_1 & 0 & \dots & 0 \\
    0 & d_2 & \dots & 0 \\
    \vdots & \vdots & \ddots & 0 \\
    0 & 0 & \dots & d_n\\
  \end{mtx}
\]

\subsubsection{Matriz invertible}
\begin{definition}
Una matriz cuadrada $A$ de orden finito $n\times n$ es invertible si existe una matriz de orden $n \times n$, que denotamos como $A^{-1}$, tal que el producto $A \cdot A^{-1} = A^{-1}\cdot A= I_n$. Siendo $I_n$ la matriz identidad de orden $n$. Llamaremos a $A^{-1}$ como la matriz inversa de $A$    
\end{definition}

\subsubsection{Matriz semejante}
\begin{definition}
Dos matrices cuadradas $A$ y $B$ de orden $n \times n$ son semejantes si existe una matriz invertible $P$ de dimensión $n \times n$ de manera que se cumpla la siguiente relación:
\[
    P^{-1}AP = B 
\]    
\end{definition}


\subsection{Autovalores y Autovectores}
\begin{definition}
Sea $A$ una matriz cuadrada de orden $n \times n$:

\begin{enumerate}
    \item Un autovector de $A$ es un vector diferente de 0 en $\mathbb{R}^n$ tal que $Av = \lambda v$ para algún escalar $\lambda$.
    \item Un autovalor de $A$ es un escalar $\lambda$ tal que la ecuación $Av = \lambda v$ tiene una solución no trivial
\end{enumerate}    
\end{definition}

\begin{prop} \label{autoli}
  Los autovectores asociados a autovalores diferentes son linealmente independientes.
\end{prop}

\begin{prop} \label{autotr}
	Una matriz cuadrada $A$ comparte autovalores con su transpuesta $A^{t}$.
\end{prop}

\begin{prop}\label{mtxeig}
   Dado $V$ un $K-$espacio vectorial y un endomorfismo $T: V \to V $ diremos que $\lambda \in K$ es un valor propio de $T$ si y sólo si existe un vector no nulo $v \in V$ tal que: 
   \[
    T(v) = \lambda I_nv
   \]
\end{prop}

  Lo que se puede representar tambien de la manera:
   \begin{gather*}
       T(v) = \lambda I_nv\\
       T(v) - \lambda I_nv= \cancel{\lambda I_nv - \lambda I_nv}_0\\
       T(v) - \lambda I_nv = 0 \\
   \end{gather*}
   
   De esta forma se obtiene la relación:
   \[
     (T - \lambda I_n)v = 0 \\
   \]


\subsubsection{Autoespacio}
\begin{definition}
  Sea $A$ una matriz cuadrada de orden $n \times n$, sea $\lambda$ un autovalor de $A$. El $\lambda$-autoespacio de $A$ es el conjunto solucion del sistema $(A-\lambda I_n)v = 0$. O lo que es lo mismo el subespacio $\opnm{Nuc}(A-\lambda I_n)$
\end{definition}

\subsubsection{Matrices Diagonalizables}
\begin{definition}
    Sea $A$ un matriz de orden $n \times n$, $A$ sera diagonalizable si es semejante a una matriz diagonal $D$, lo que implica que existe una matriz cuadrada invertible $P$ tal que se cumple la siguiente relación:
    \[
        P^{-1}AP = D
    \]
\end{definition}

\begin{prop}
  Sea $A$ una matriz diagonalizable de orden $n \times n$ con matriz daigonal asociadada $D$ y matriz de paso $P$, además sean $\lambda_{1},\lambda_{2},\dots,\lambda_{n}$ autovalores de $A$ y $v_{1},v_{2},\dots,v_{n}$ autovectores de $A$, entonces se cumplirá:
	\begin{itemize}
		\item $D = \opnm{Diag}(\lambda_{1},\lambda_{2},\dots,\lambda_{n})$
		\item $P = \begin{mtx}{4}
				\vdots & \vdots & \vdots & \vdots\\
				v_{1} & v_{2} & \dots &v_{n}\\
				\vdots & \vdots & \vdots & \vdots\\
		\end{mtx}$
	\end{itemize}
\end{prop}

\begin{prop}\label{diagpot}
	Sea $A$ una matriz diagonalizable de orden $n \times n$ con matriz diagonal asociada $D$ y matriz de paso $P$, entonces se cumplirá:
	\[
	  A^{n} = PD^{n}P^{-1}
	\]
\end{prop}


\subsection{Procesos de Markov}

\begin{definition}
    Un proceso aleatorio $\{X(t)|t\in T\}$ es llamado un proceso de Markov si para cualquier momento $t_0<t_1<\dots<t_n<t$ la distribución condicional en $X(t)$ dados los valores de $X(t_0),\dots,X(t_n)$ este depende únicamente de $X(t)$. Formalmente:
    \[
        P[X(t)\leq x| X(t_n) = x_n,\dots,X(t_0) = x_0] = P[X(t)=x| X(t_n) = x_n]
    \]
\end{definition}

\subsubsection{Matriz estocástica}

\begin{definition}
    Se llama matriz estocástica a una matriz cuadrada cuyos elementos son todos positivos y la suma de los elementos de cada columna es igual a 1.
    \[
        \mathcal{P} = \left[
        \begin{array}{cccc}
            p_{11} & p_{12} & \dots & p_{1n}\\
            p_{21} & p_{22} & \dots & p_{2n}\\
            \vdots & \vdots & \ddots & \vdots\\
            p_{n1} & p_{n2} & \dots & p_{nn}
        \end{array}
        \right] \; p_{ij} \geq 0 \land \sum_{j=1}^{n}p_{ij} = 1 \quad \forall i = 1,2,\dots,n
    \] 
\end{definition}

\begin{theorem} \label{esto1}
	Sea $A$ una matriz estocástica de orden $n \times n$, entonces $\lambda = 1$ es un autovalor de $A$.
\end{theorem}

\begin{proof}
	Sea $a = (1,1,\dots,1)^{t}$ un vector columna $n$-dimensional con todos sus componentes iguales a 1. Además sabemos por la proposición \ref{autotr}  que $A$ compartira autovalores con su transpuesta.

	entonces efectuaremos el producto $A^{t}a$:
	\[
		\begin{mtx}{4}
			p_{11} & p_{21} & \dots & p_{n1}\\
			p_{12} & p_{22} & \dots & p_{n2}\\
			\vdots & \vdots & \ddots & \vdots\\
			p_{1n} & p_{2n} & \dots & p_{nn}
		\end{mtx}
		\begin{mtx}{1}
		  1\\
			1\\
			\vdots\\
			1\\
		\end{mtx} =
		\begin{mtx}{1}
		  \sum_{i = 1}^{n}p_{i1}\\
		  \sum_{i = 1}^{n}p_{i2}\\
			\vdots\\
		  \sum_{i = 1}^{n}p_{in}\\
		\end{mtx}
		=1
		\begin{mtx}{1}
		  1\\
			1\\
			\vdots\\
			1\\
		\end{mtx}
	\]
	Finalmente observamos que 1 es un autovalor de $A^{t}$ y por propiedad de $A$, además tiene como autovector asociado a $a$.
\end{proof}


\begin{theorem}[Teorema de Perron-Frobenious.] \label{perron}
	Sea A una matriz de orden $n$ no nula y suponiendo que $a_{ij} > 0$, entonces $A$ tiene un autovalor real o positivo mayor que el módulo de todos los demás y tal que tiene un autovector asociado con todas sus componentes positivas.
\end{theorem}

\subsection{Modelos científicos}

La realidad posee diferentes sistemas complejos los cuales si analizamos de una manera general se presenta una dificultad para poder observar las leyes que gobiernan su funcionamiento. Los modelos científicos son una de las herramientas para poder representar estos fenómenos de una manera mas sistemática.
El modelamiento científico se realiza a través de abstraer los diferentes procesos que afectan a un sistema con el fin de analizar los rasgos mas significativos de este.

% section Marco Teórico (end)

\section{Sistemas de evolución}
\subsection{Definición}

\subsection{Antecedentes}

\subsection{Áreas de aplicación}




Una aplicación del concepto de autovalores y autovectores se puede encontrar en el estudio de poblaciones. Esto es posible a través de la construcción de modelos biológicos que pueden ser resueltos a través de conceptos algebraicos. 
La modelización de los sistemas biológicos parte de tomar en cuenta una de las características fundamentales del ciclo biológico en el cual los organismos vivos nacen, se reproducen y mueren lo que en consecuencia causa una variación en la cantidad de individuos. Además la población de una especie que habita un ecosistema se ve afectada por la interacción con otras especies, principalmente por la relación depredador-presa.




\section{Modelos biológicos discretos} % (fold)
\label{sec:Modelos biologicos discretos}

Los modelos biológicos discretos reciben su nombre debido a que solo se considera el estado de un sistema en un conjunto discreto de instantes de tiempo, lo que quiere decir que solo estudian el sistema en momentos con una separación similar entre cada periodo.
Estos modelos suelen definidos por ser representados a 
de un sistema de ecuaciones recursivas, en el cual una ecuación o formula se aplica de manera recurrente sobre el sistema y de esta manera se obtiene el estado del sistema en un periodo de tiempo determinado el cual se encuentra determinado por los periodos anteriores de tiempo. Estos modelos han presentado una gran variedad de aplicaciones en diversos campos, uno de ellos siendo el estudio de poblaciones.

\subsection{Modelos multidimensionales lineales}

En un modelo multidimensional la población se encuentra divida en diferentes agrupaciones, las cuales pueden estar definidas a razón de alguna característica en concreto del individuo, como podría ser el caso de la edad, las características vitales,la capacidad reproductiva u otros. Para lo cual se usara una variable diferente para denotar a cada grupo de la población la cual representa la cantidad de individuos que pertenecen a dicho grupo en un periodo de tiempo determinado.

El desarrollo que presentara cada grupo del sistema estará definido por la ecuación recursiva con la cual es posible obtener el numero de individuos en un instante de tiempo $t$ a partir de la cantidad de individuos en cada uno de los grupo en el instante anterior $t-1$. Por lo tanto obtendremos una ecuación recursiva para cada grupo obteniendo así un sistema de ecuaciones recursivas que definen la evolución del sistema en su totalidad. 

% subsubsection Modelos biologicos discretos (end)

\subsection{Cadenas de Markov} % (fold)
\label{sub:Cadenas de Markov}
Una cadena de Markov es una serie de eventos en donde la probabilidad de que un evento suceda está determinado únicamente por el evento inmediato anterior. Las cadenas de Markov son un tipo de proceso estocástico estacionario lo que quiere decir que a diferencia de un proceso estocástico convencional, que suele ser entendido como impredecible, en este podemos encontrar una serie de características que lo hacen hasta cierto punto predecible. 

Para ilustrar este apartado su relacion con los modelos multidimensionales lineales imaginemos una población con diferentes estados en el cual el número de individuos depende de la cantidad de individuos que hubieron en cada estado en un periodo de tiempo anterior. Por ejemplo tomemos los estados $E_1, E_2,...,E_n$. La cantidad de individuos en un determinado estado en el periodo $t+1$ estará determinado únicamente por los individuos existentes en cada estado en el periodo $t$, siendo que existe una posibilidad de cambiar desde el estado $E_j$ al estado $E_i$ la cual estará denotada por $p_{ij}$, por lo que tendremos que $p_{1j} + p_{2j} + \dots + p_{nj} = 1$ para $i = 1,2,...n$. Por lo que tendremos que el numero de individuos en el periodo $t+1$ ($\{X_i(t+1)\}^n_{i=1}$), se puede representar en función del numero de individuos en el tiempo $t$ (${X_i(t)}^n_{i=1}$), mediante el siguiente sistema de ecuaciones:

\[
	\begin{cases}
	 X_{1}(t+1) = p_{11}X_{1} + p_{12}X_{2} + \dots + p_{1n}X_{n}\\
	 X_{2}(t+1) = p_{21}X_{1} + p_{22}X_{2} + \dots + p_{2n}X_{n}\\
	 \vdots\\
	 X_{n}(t+1) = p_{n1}X_{1} + p_{n2}X_{2} + \dots + p_{nn}X_{n}\\
	\end{cases}
\]

%\pagebreak

Este sistema se puede representar en forma matricial como:

\[
  \vec{X}(t+1) = A\vec{X}(t)
\]

donde $\vec{X}(t) = 
\left[\begin{array}{c} X_{1}(t)\\ X_{2}(t)\\ \vdots\\ X_{n}(t)\\ \end{array}\right]$ ,
$\vec{X}(t+1) = \left[\begin{array}{c} X_{1}(t+1)\\ X_{2}(t+1)\\ \vdots\\ X_{n}(t+1)\\ \end{array}\right]$ y 
$A = \left[\begin{array}{cccc}
		p_{11}& p_{12}& \dots & p_{1n}\\
		p_{21}& p_{22}& \dots & p_{2n}\\
		\vdots & & \ddots & \vdots\\
		p_{n1}& p_{n2}& \dots & p_{nn}\\
\end{array}\right]$,

además los elementos de cada columna de la matriz $A$ suman 1 y todos los elementos serán positivos, por lo tanto observamos que la matriz $A$ es una matriz estocástica.

De este modo si se desea conocer la distribución de la población transcurridas $m$ unidades de tiempo desde el tiempo $t$, tendremos $\vec{X}(t+m) = A\vec{X}(t+m-1) = A^{2}\vec{X}(t+m-2) = \dots = A^{m}\vec{X}(t)$. A este tipo de sistemas se le conoce como cadenas de Markov, siendo aqui donde podemos notar que la matriz $A$ representa el conjunto de funciones recursivas que se menciono en la sección anterior, por lo cual una cadena de Markov es una forma de representar un modelo multidimensional lineal.

Luego para resolver sistemas como este se simplificará la matriz  $A$ haciendo uso de la matriz diagonal asociada $D$ y la matriz de paso $P$ la cual esta formada por los autovectores de la matriz $A$. Para luego realizar un análisis del sistema usando la propiedad de las matrices diagonalizables: $A^{m} = PD^{m}P^{-1}$.

% subsection Cadenas de Markov (end)

\subsection{Modelos de Leslie} % (fold)
\label{sec:Modelos de Leslie}

Este modelamiento se usa principalmente para el estudio de la distribución de la población en distintas etapas del ciclo vital. Esta división suele hacerse por edades pero también es aplicable a diferentes estadios o tamaños del espécimen. 

Para lo cual la distribución de una población estará determinada por el vector:
\[
	\vec{N}(t) = \left[\begin{array}{c} N_{0}(t)\\ N_{1}(t)\\ \vdots\\ N_{k-1}(t)\\ \end{array}\right]
\]
que simplifica la cantidad de individuos en las $k$ etapas de desarrollo $N_0,N_1,\dots,N_{k-1}$.

Luego buscamos obtener un sistema de ecuaciones recursivas que representen la evolución del sistema en el tiempo, lo que quiere decir que se busca obtener el vector de distribución $\vec{N}(t+1)$ a partir de los datos conocidos de $N(t)$, para lo cual es necesario conocer los diferentes factores que contribuyen a la evolución de una población.

Para ejemplificar este apartado supondremos que se está estudiando una población agrupada por edades de modo que el grupo de edad $i$ engloba a los individuos cuyas edades se encuentran en el intervalo $\left[i,i+1\right[$. Además, se tendrán en cuenta las siguientes características del sistema:

\begin{itemize}
    \item La población se encuentra aislada, por lo tanto no se presentaran movimientos migratorios que afecten a los distintos grupos de edad.
    \item El sistema cuenta con los recursos suficientes para soportar el crecimiento que se presenta en la población.
    \item Si la población presenta una reproducción sexual, el modelo se construirá solo tomando en cuenta a la población de hembras de la especie puesto que a partir de esta despoblación se podrá estimar la población total.
\end{itemize}

Luego para realizar el planteamiento del modelo se debe de conocer:

\begin{itemize}
    \item La tasa de supervivencia de cada uno de los $k$ grupos de edad, a la cual denotaremos $s_i$ con $i=1,\dots,k-1$. La tasa de supervivencia del grupo $i$ puedes ser descrita como la proporción de los individuos del grupo $i$ que han sobrevivido luego de un periodo de tiempo $t$ y que por lo tanto pertenecerán ahora al grupo $i+1$. Puesto que tenemos como grupo final al grupo de edad $k-1$ podemos deducir que $s_{k-1} = 0$. Además, se cumplirá que $0 \leq s_i \leq 1$ para todo $i$ puesto que estas tasas representan a las proporciones de supervivencia.

    \item La tasa de fertilidad que denotamos como $f_i$ con $i = 1,\dots,k-1$. La tasa de fertilidad $f_i$ representa el número de descendientes que tiene un individuo del grupo $i$ en cada periodo temporal. En este caso podemos asegurar que $f_i \geq 0$ para todo $i$, puesto que representa el número de individuos nacido de un determinado grupo de edad por cada individuo en dicho grupo.
\end{itemize}

Luego de delimitadas las condiciones y haber definido los factores que se tomaran en cuenta, es posible plantear el sistema de ecuaciones que permite describir a la población en el instante de tiempo $t+1$, conociendo previamente la distribución de la población en el instante $t$:

\[
  \begin{cases}
    N_{0}(t+1) = f_{0}N_{0}(t) +  f_{1}N_{1}(t) + \dots + f_{k-1}N_{k-1}(t)\\
    N_{1}(t+1) = s_{0}N_{0}(t)\\
		\vdots\\
		N_{k-1}(t+1) = s_{k-2}N_{k-2}(t)\\
	\end{cases}
\]
%\pagebreak

El cual se puede representar de la forma matricial:

\[
  \vec{N}(t+1) = L \vec{N}(t) \text{ con } L = 
\left[\begin{array}{cccc}
		f_{0}& f_{1}& \dots & f_{k-1}\\
		s_{0}& 0 & \dots & 0\\
		\vdots & \ddots & \ddots & \vdots\\
		0 & \dots & s_{k-2} & 0\\
\end{array}\right]
\]

A la matriz $L$ se le conoce como matriz de Leslie, en honor al fisiólogo Patrick Holt Leslie, que fue el primero en plantear este tipo de problemas en 1945, y el modelo presentado se le conoce como modelo de Leslie.


De este modo si queremos conocer la distribución de la población luego de $n$ periodos de tiempo desde un instante inicial $t$ tendremos la siguiente serie $\vec{N}(t+n) = L\vec{N}(t+n-1) = \dots = L^{n}\vec{N}(t) $. Para la resolución de este tipo de problemas se hace uso de la simplificación de la matriz $L$ hallando sus autovalores y autovectores, con los cuales se encontrará la matriz diagonal $D$ asociada a L y su correspondiente matriz de paso $P$.

Asi se tiene que:

\[
  \vec{N}(t+n) = L^{n}\vec{N}(t) = PD^{n}P^{-1}(t)
\]
Luego por el teorema de Perron-Frobenious es posible demostrar que una matriz de Leslie como lo es  $L$ poseerá un autovalor real positivo simple $\lambda_d$, el cual sera mayor que todos los demás autovalores, por lo cual se le denomina como autovalor dominante. Además, el autovector $\vec{u}_d$ asociado a dicho autovalor tiene todas sus componentes positivas.

Por lo tanto, al estudiar el comportamiento a largo plazo de una población de Leslie se observara que la proporción de los individuos entre los distintos grupos de edad $N_0,N_1,\dots,N_k$ a largo plazo vendrá dada por la proporción entre las componentes del autovector $\vec{u}_d$ asociado al autovalor dominante.

Luego estudiando al autovalor dominante ($\lambda_d$) se cumplirán las siguientes propiedades:

\begin{itemize}
	\item Si $\lambda_{d}> 1$ la población crecerá a lo largo del tiempo.
	\item Si $\lambda_{d} = 1$ la población sera estable.
	\item Si $\lambda_{d}< 1$ la población decrecerá a lo largo del tiempo y terminará extinguiéndose.
    \item El porcentaje de crecimiento o decrecimiento de la población  luego de cada  periodo de tiempo vendrá determinada por $(1-\lambda_d) \times 100\% $
\end{itemize}

Sabiendo que la población será estable cuando $\lambda_d = 1$, se puede demostrar que esta condición se cumplirá cuando $f_0+s_0f_1+s_0s_1f_2+\dots+s_{k-2}f_{k-1} = 1$. La expresión $R = f_0+s_0f_1+s_0s_1f_2+\dots+s_{k-2}f_{k-1}$ se denomina tasa neta de reproducción y representa el número promedio de crías hembra que tiene una hembra durante su esperanza de vida. 

\subsection{Modelización discreta de ecosistemas}

Se puede describir un modelo biológico discreto que presenta una mayor complejidad, se trata del modelamiento discreto de ecosistemas, en el cual se analiza la interacción de entre las distintas poblaciones que conforman un ecosistema siendo que se conoce como afecta en promedio la presencia de una población en un periodo de tiempo a las otras poblaciones que residen en el ecosistema.

De manera análoga a los demás modelos presentados, el numero de individuos de una población ($y_i$) en un periodo de tiempo $t+1$ estará determinado por el numero de individuos que existían de esa población y de otras en un tiempo $t$ por lo tanto el sistema puede ser representado usando el siguiente sistema de ecuaciones recursivas:

\[
	\begin{cases}
	 y_{1}(t+1) = r_{11}y_{1} + r_{12}y_{2} + \dots + r_{1k}y_{k}\\
	 y_{2}(t+1) = r_{21}y_{1} + r_{22}y_{2} + \dots + r_{2k}y_{k}\\
	 \vdots\\
	 y_{k}(t+1) = r_{k1}y_{1} + r_{k2}y_{2} + \dots + r_{kk}y_{k}\\
	\end{cases}
\]

De donde obtenemos $\vec{Y}(t+1) = R\vec{Y}(t)$ con 
$Y = \left[\begin{array}{c}
    y_1\\
    y_2\\
    \vdots\\
    y_k\\
\end{array}\right]$ y
$R = \left[\begin{array}{cccc}
    r_{11} & r_{12} & \dots & r_{1k} \\
    r_{21} & r_{12} & \dots & r_{2k} \\
    \vdots & \vdots & \ddots & \vdots \\
    r_{k1} & r_{k2} & \dots & r_{kk} \\
\end{array}\right]$

Por lo tanto si se desea conocer la distribución entre las distintas poblaciones luego de $n$ periodos de tiempo desde un instante $t$ se tendrá que $\vec{Y}(t+n) = R\vec{Y}(t+n-1) = \dots = R^{n}\vec{Y}(t)$, de donde si se tiene que $R$ es diagonalizable el análisis del sistema se simplificara usando la matriz diagonal $D$ asociada a $R$ junto con la correspondiente matriz de paso $P$:
\[
    \vec{Y}(t+n) = R^{n}\vec{Y}(t) = PD^{n}P^{-1}(t)
\]

Si además de las interacciones entre especies existe algun otro factor que describa una variación cuantitativa de las poblaciones presentes en el ecosistema independientemente de el numero de individuos que existan en un sistema, se tendrá que:

\[
	\begin{cases}
	 y_{1}(t+1) = r_{11}y_{1} + r_{12}y_{2} + \dots + r_{1k}y_{k} + f_1(t+1)\\
	 y_{2}(t+1) = r_{21}y_{1} + r_{22}y_{2} + \dots + r_{2k}y_{k} + f_2(t+1)\\
	 \vdots\\
	 y_{k}(t+1) = r_{k1}y_{1} + r_{k2}y_{2} + \dots + r_{kk}y_{k} + f_k(t+1)\\
	\end{cases}
\]

Lo que luego representamos como $\vec{Y}(t+1) = R\vec{Y}(t) + \vec{F}(t+1)$ con  
$\vec{F} = \left[\begin{array}{c}
    f_1(t+1)\\
    f_2(t+1)\\
    \vdots\\
    f_k(t+1)\\
\end{array}\right]$
\include{"sections/modelos-biologicos-continuos.tex"}
\section{Ejemplos de aplicación}

\subsection{Caso contagio de enfermedad}
Se busca analizar el desarrollo de una cepa de una enfermedad contagiosa en una población de 1000 individuos. Se conoce que luego de una semana un individuo sano posee un 70\% de probabilidades de contraer la enfermedad, además se conoce que luego de una semana un individuo enfermo tiene un 40\% de probabilidades de curarse sin obtener inmunidad a la enfermedad, un 10\% de probabilidad de fallecer a causa de la enfermedad y un 40\% de probabilidad de desarrollar inmunidad permanente a la enfermedad. Se busca conocer la distribución de la población luego de cinco semanas de haber comenzado el contagio.

Comenzaremos el análisis observando los posibles estados en los que se puede encontrar un individuo:

\begin{itemize}
    \item Sano
    \item Enfermo
    \item Inmunizado
    \item Muerto
\end{itemize}

Además se tiene que el vector de población inicial sera igual a:

\[
    \vec{p} = 
    \left[
    \begin{array}{c}
        1000\\
        0\\
        0\\
        0
    \end{array}
    \right]
\]

Conocemos las probabilidades que definen los cambios de estado de un ejemplar sano y enfermo. Los estados restantes no presentaran una variación en el tiempo. 



Por lo tanto el modelo del desarrollo de la enfermedad sera de la siguiente manera:

\begin{figure}[htb]
\centering
\begin{tikzpicture}
\node[state] (s1) {Sano};
\node[state, right of=s1] (s2) {Enfermo};
\node[state, below left of=s2] (s3) {Inmune};
\node[state, below right of=s2] (s4) {Muerto};

\draw (s1) edge[loop left] node {0.3}  (s1);
\draw (s1) edge[bend left] node {0.7}  (s2);


\draw (s2) edge[bend left=12] node {0.4} (s1);
\draw (s2) edge[loop right] node {0.1} (s2);
\draw (s2) edge[bend right=12] node{0.4} (s3);
\draw (s2) edge[bend left=12] node{0.1} (s4);

\draw (s3) edge[loop left] node {1} (s3);
\draw (s4) edge[loop right] node{1} (s4);

\end{tikzpicture}
\caption{Cadena de Markov}

\end{figure}

De donde podemos obtener la siguiente matriz de transición:

\[
    A = 
    \left[
        \begin{array}{cccc}
            0.3 & 0.4 & 0 & 0 \\
            0.7 & 0.1 & 0 & 0 \\
            0 & 0.4 & 1 & 0 \\
            0 & 0.1 & 0 & 1
        \end{array}
    \right]
\]
Podemos observar que la matriz solo posee valores reales positivos y además los elementos de cada columna suman 1 entonces se puede decir que la matriz de transición es una matriz estocástica. Entonces por el teorema \ref{esto1} podemos garantizar que $\lambda = 1$ será un autovalor de la matriz.

Con ayuda de una herramienta computacional se obtienen los autovalores y autovectores asociados a la matriz de transición:

\begin{gather*}
    \lambda_1,\lambda_2 = 1 \quad E_{\lambda_1,\lambda_2} = L\{(0,0,1,0)^t,(0,0,0,1)^t\}\\
    \lambda_3 = 0.738516 \quad E_{\lambda_3} = L\{(-2.38516,-2.6148,4,1)^t\}\\
    \lambda_4 = -0.338516 \quad E_{\lambda_4} = L\{(8.38516,-13.3852,4,1)^t\}\\
\end{gather*}

Por la proposición \ref{autoli} se sabe que los autovectores asociados a autovalores diferentes serán linealmente independientes por lo tanto el conjunto $B = \{v_1,v_2,v_3,v_4\}$ sera una base de espacio vectorial $\mathbb{R}^4$ donde $v_i$ sera el autovector asociado al autovalor $\lambda_i, i = 1,2,3,4$

Como la multiplicidad algebraica de cada autovalor coincide con la multiplicidad geométrica. Entonces la matriz $A$ sera diagonalizable de la siguiente forma:

\[
    A = PDP^{-1}
\]

Luego si se quiere conocer la distribución de la población luego de 5 semanas se hará uso del teorema \ref{diagpot} de las matrices diagonalizables:

\[
    A^5\vec{p} = PD^5P^{-1}\vec{p}
\]

Como el vector de población inicial pertenece a $R^4$ este podrá ser escrito de la siguiente forma:

\[
    \vec{p} = w_1v_1 + w_2v_2 + w_3v_3 + w_4v_4
\]
 y la matriz $P^{-1}$ describirá una matriz de cambio de base de la base canónica a la base $B$. Entonces podemos desarrollar la expresión $A^5\vec{p} = PD^5P^{-1}\vec{p}$ de  la siguiente manera

 \begin{gather*}
    A^5\vec{p} = P D^5 \left[\begin{array}{c} w_1\\ w_2\\ w_3\\ w_4 \end{array}\right]\\
    A^5\vec{p} = P \left[\begin{array}{c} \lambda_1^5w_1\\ \lambda_2^5w_2\\ \lambda_3^5w_3\\ \lambda_4^5w_4 \end{array}\right]\\
    A^5\vec{p} = \lambda_1^5w_1v_1 + \lambda_2^5w_2v_2 + \lambda_3^5w_3v_3 + \lambda_4^5w_4v_4\\
 \end{gather*}

 Hallando los valores $w_i$:

 \[
    \vec{p} = 800v_1 + 200v_2 - 248.558 v_3 + 48.556 v_4
 \]

 Finalmente la expresión $A^5\vec{p}$ queda de la siguiente forma:

 \[
        A^5\vec{p} = 1^5(800)v_1 + 1^5(200)v_2 + 0.738516^5(-248.558)v_3 -0.338516^5(48.556)v_4
 \]

 Opererando:

 \[
    A^5\vec{p} = (0,0,800,0) + (0,0,0,200) + (130,142,-218,-54) + (-2,3-1,0)\\
\]


Obtenemos que luego de 5 semanas la población presentara la siguiente distribución:
\[
    A^5\vec{p} = (128,145,581,146)
\]
 teniendo a 128 individuos sanos, 145 enfermos,581 que han desarrollado inmunidad y 146 que han perecido a causa de la enfermedad. Sumando los valores obtenemos 1000 que coincide con el numero de individuos en la población.

 Adicionalmente podemos hacer el estudio en un tiempo $n$ que tiende al infinito tal que:
\[
    \lim_{n \to \infty}A^n\vec{p} = \lim_{n \to \infty} \underbrace{1^n}_{=1}(800)v_1 + \underbrace{1^n}_{=1}(200)v_2 + \underbrace{0.738516^n}_{\approx 0}(-248.558)v_3 \underbrace{-0.338516^n}_{\approx 0}(48.556)v_4
\]
De donde obtenemos que:
\[
    A^n\vec{p} = (0,0,800,200)
\]
que quiere decir que cuando la enfermedad ya no se pueda seguir propagando habrán perecido 200 individuos a causa de esta y habrán aproximadamente 800 individuos con inmunidad.
 
\subsection{Caso crecimiento poblacional de ratones}
Se estudia a una especie de ratones donde se consideran 3 etapas vitales: crías, jóvenes y adultos. Esta especie de ratón progresa de cría a joven en un determinado periodo de tiempo, de joven a adulto en un periodo de tiempo similar y, finalmente, los jóvenes y adultos tienen crías en un periodo de tiempo similar a los anteriores. Se desea saber la distribución de la población luego de 10 periodos de tiempo, con una distribución inicial de 10 adultos, 15 jóvenes y 40 crías. Se conoce que solo el 20\% de las crías llegan a ser jovenes y solo un 50\% de los ratones jóvenes llegan a adultos, cada ratón joven tiene una media de 5 crías y que cada adulto tiene una media  3 de crías. 

Lo primero que debemos hacer es definir los posibles etapas de un individuo de la especie, en este caso se nos otorga de manera directa los cuales son:

\begin{itemize}
    \item Cría (C)
    \item Joven (J)
    \item Adulto (A)
\end{itemize}

además se tiene un vector de población inicial $\vec{p} = \left[ \begin{array}{c}
    40 \\
    15 \\
    10
\end{array} \right] $

Denotamos las siguientes expresiones:

\begin{align*}
    C_t &= \text{numero de crías en el instante de tiempo t}\\
    J_t &= \text{numero de jóvenes en el instante de tiempo t}\\
    A_t &= \text{numero de adultos en el instante de tiempo t}
\end{align*}

Luego definimos relaciones entre las diferentes etapas:
\[
\begin{cases}
    C_t = 5 J_{t-1} + 3 A_{t-1} \text{(Cada adulto tiene 3 crías y cada joven 5 crías)}\\
    J_t = 0.2 C_{t-1} \text{(20\% de las crías llegan a la juventud)}\\
    A_t = 0.5 L_{t-1} \text{(50\% de los jóvenes llegan a ser adultos)}\\
\end{cases}
\]

Por lo tanto podemos representar gráficamente al modelo de la siguiente forma:
\begin{figure}[htb]

\centering
\begin{tikzpicture}
\node[state] (s1) {C};
\node[state, right of=s1] (s2) {J};
\node[state, right of=s2] (s3) {A};

\draw (s1) edge[bend left] node {0.2}  (s2);
\draw (s2) edge[bend left] node {0.5}  (s3);


\draw (s2) edge[bend left=12] node {5} (s1);
\draw (s3) edge[bend left] node {3} (s1);

\end{tikzpicture}

\caption{Modelo de Leslie}
\end{figure}

Además se obtiene la matriz de transición ($A$):

\[
A =
\left[
    \begin{array}{ccc}
        0 & 5 & 3 \\
        0.2 & 0 & 0 \\
        0 & 0.5 & 0
    \end{array}
\right]
\]

Se sabe que la distribución de la población en el instante de tiempo $t$ es dependiente del instante de tiempo anterior $t-1$:

\[
    \left[
    \begin{array}{c}
        C_t \\
        J_t \\
        A_t
    \end{array}
    \right] = 
    \left[
    \begin{array}{ccc}
        0 & 5 & 3 \\
        0.2 & 0 & 0 \\
        0 & 0.5 & 0
    \end{array}
    \right]
    \left[
    \begin{array}{c}
        C_{t-1} \\
        J_{t-1} \\
        A_{t-1}
    \end{array}
    \right]
\]

Hallamos los autovalores y autovectores de la matriz $A$:

\begin{gather*}
    \lambda_1 = -0.3389; \quad v_1 = (1.149,-0.678,1)\\
    \lambda_2 = -0.7865; \quad v_2 = (6.186,-1.573,1)\\
    \lambda_3 = 1.1254; \quad v_3 = (12.666,2.251,1)\\
\end{gather*}

Como $v_1, v_2 \text{ y } v_3$ están asociados a tres autovalores distintos entonces son linealmente independientes y conforman una base $B$ del espacio $\mathbb{R}^3$.

Hallando los coeficientes de la combinación lineal de $\vec{p}$ en la base $B$:
\[
    \vec{p} = 11.281v_1 - 6.677v_2 + 5.396v_3
\]

Realizando el procedimiento visto en el anterior ejercicio obtenemos la siguiente expresión con la cual determinamos $A^t\vec{p}$:

\[
    A^t\vec{p} = (-0.3389)^t (11.281)v_1 + (-0.7865)^t(-6.677) v_2 + (1.1254)^t(5.396)v_3
\]

Como deseamos obtener la distribución luego de 10 periodos, reemplazamos $t$ con 10:


\begin{gather*}
   A^{10}\vec{p} = \underbrace{(-0.3389)^{10}}_{\approx 0} (11.281)v_1 + \underbrace{(-0.7865)^{10}}_{\approx 0}(-6.677) v_2 + (1.1254)^{10}(5.396) v_3\\
   A^{10}\vec{p} = (17.5849)v_3\\
\end{gather*}


Operando tenemos:

\[
\vec(p)_{10}  = 
\left[\begin{array}{c}
    223 \\
    36 \\
    18
\end{array}\right]
\]

Siendo $\vec{p}_{10}$ la representación vectorial de la distribución en el decimo periodo. Ademas podemos obtener la tasa de crecimiento de un periodo a traves de la expresión  \mbox{$|1 - \lambda_{d}|\times 100\%$} siendo esta cercana a $12.5\%$

Finalmente para un tiempo $t$ que tiende la infinito:

\[
	\lim_{t \to \infty} A^tE = \lim_{t \to \infty}\underbrace{\lambda_1^t}_{\approx 0} w_1v_1 + \underbrace{\lambda_2^t}_{\approx 0} w_2v_2+  \lambda^t w_3v_3
\]

En el caso $t \to \infty$ la distribución esta definida solo por

\[
    A^tE = \lambda^t_3 w_3v_3
\]

    
%\subsection{Caso Ecosistema}


% Bibliografia

\begin{thebibliography}{6}
\bibitem{texbook}
Jose D. Zavala (2021) \emph{Teorema de Perron - Frobenious}, Universidad Nacional de San Agustin de Arequipa.

\bibitem{lamport94}
Margarita Arias, Juan campos (2011) \emph{Jugando con matrices positivas: eficiencia de un estado incial.: a document preparation system}, Universidad de Granada.

\bibitem{key}
Antonio Diéguez (2013) \emph{The explanatory function of models in biology}, Universidad de Málaga

\bibitem{wan1}
Frederick Y.M. Wan \emph{Discrete Space-time Models in Biology}, University of California

\bibitem{alvarez}
R. Álvarez-Nodarse \emph{Modelos matemáticos en bíologia:  un viaje de ida y vuelta}, Universidad de Sevilla
\end{thebibliography}



\end{document}
